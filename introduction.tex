% $Id: introduction.tex 65669 2015-01-09 14:55:20Z tgershon $

\section{Introduction}
\label{sec:Introduction}
%LHCb-ANA-2014-039-20140923
The interference between $\Bs$ meson decay amplitudes to $\jpsi X$ $\CP$ eigenstates directly or via mixing gives rise to a measurable $\CP$-violating phase $\phis$. In the Standard Model (SM), for $\bquark\rightarrow\cquark\cquarkbar\squark$ transitions and ignoring subleading penguin contributions, this phase is predicted to be $-2\betas$, where $\betas = \arg[−(\Vts\Vtbs)/(\Vcs\Vcbs)]$ and $V_{ij}$ are the elements of CKM quark flavour mixing matrix~\cite{Kobayashi:1973fv, Cabibbo:1963yz}. The indirect determination via global fits to experimental data gives $2\betas$ = 0.0364$\pm$0.0016~$\rad$~\cite{Charles:2011va}. This precise indirect determination within the SM makes the measurement of $\phis$ interesting since new physics (NP) processes could modify the phase if new particles were to contribute to the $\Bs-\Bsb$ box diagrams~\cite{Buras:2009if, Chiang:2009ev}.

The direct measurements of $\phis$ using $\Bs\to\jpsi\phi$ decays with $\jpsi\to\mumu$ were made by the $\lhcb$ \cite{Aaij:2014zsa}, $\atlas$ \cite{Aad:2014cqa} and $\cms$ \cite{Khachatryan:2015nza} experiments at the $\lhc$ and the $\cdf$ \cite{Aaltonen:2012ie} and $\dzero$ \cite{Abazov:2011ry} experiments at the $\tevatron$. The $\lhcb$ collaboration has reported the world's best measurement of $\phis=-0.010\pm0.039$~$\rad$ \cite{Aaij:2014zsa} as a combined result from $\Bs\to\jpsi\Kp\Km$ and $\Bs\to\jpsi\pip\pim$ decays based on 3~$\invfb$ of integrated luminosity. So far, all experimental results are in agreement with the SM predictions. The precision of the results can be increased by including additional decay modes. Such results can be combined with those from the $\Bs\to\jpsi\Kp\Km$ and $\Bs\to\jpsi\pip\pim$ decays \cite{Aaij:2013oba, Aaij:2012-067}. 

This analysis presents the $\phis$ measurement using a tagged flavour time dependent angular analysis of $\Bs\rightarrow\jpsi(\epem)\phi(\Kp\Km)$ channel with 3~$\invfb$ of integrated luminosity collected by LHCb detector in $pp$ collisions at a centre-of-mass energy of 7~$\tev$ in 2011 and 8~$\tev$ in 2012 at the $\lhc$. The $\Bs\rightarrow\jpsi(\epem)\phi(\Kp\Km)$ channel brings about 10~$\%$ of the $\mumu$ mode statistics and is an important verification of the golden channel, $\Bs\rightarrow\jpsi(\mumu)\Kp\Km$, as kinematics for both channels are identical but sources of systematic uncertainties are different. In addition, the measurements of the decay width difference of the light (L) and heavy (H) $\Bs$ mass eigenstates, $\DGs\equiv\GL-\GH$, and the average $\Bs$ decay width, $\Gs\equiv(\GL+\GH)/2$, are presented. 

This note is structured as follows. Sec.~\ref{sec:Phenom} describes the phenomenology of the $\BsToJPsiPhi$ decay, in particular introducing the polarization dependent notation. Sec.~\ref{sec:Data} describes the data and simulation samples which are used in the analysis, including trigger and offline selections. The two types of Boosted Decision Trees are used in the selection and they are described in Sec.~\ref{subsec:BDT}. The reweighting of Monte Carlo samples is discussed in Sec.~\ref{subsec:MCreweight}. The fit of the $\Bs$ mass distribution is given in Sec.~\ref{sec:BsFit}. Sec.~\ref{sec:TimeRes} describes the procedure for extracting and calibrating the decay time resolution model. Sec.~\ref{sec:TimeAcc} describes the dependence of the efficiency as a function of decay time caused by the trigger and offline selections. The same techniques as described in Ref.~\cite{Aaij:2013oba} are used to determine decay time acceptance from data and simulation. The angular acceptance is derived from simulation, as described in Sec.~\ref{sec:AngAcc}. The effect of angular resolution is described in Sec.~\ref{sec:AngRes}. The flavour tagging algorithms and their calibration are documented in Sec.~\ref{sec:FlavTagg}. %The effective coupling factor of $S$-wave and $P$-wave components is discussed in Sec. 10. 
Sec.~\ref{sec:Results} describes the final results for the $\Bs\to\jpsi(\epem)\phi$ analysis. The sources of systematic uncertainties are described and their size estimated in Sec.~\ref{sec:SystUnc}. Conclusions are given in Sec.~\ref{sec:Concl}. Additional details of parameters for the fits, decay time resolution parameterisation, analysis code and background systematics are described in the Appendices.

If there is $\Bs\to\jpsi\phi$ decay it means that $\jpsi\to\epem$, otherwise it is indicated.


% This is the template for typesetting LHCb notes and journal papers.
% It should be used for any document in LHCb~\cite{Alves:2008zz} that is to be
% publicly available. The format should be used for uploading to
% preprint servers and only afterwards should specific typesetting
% required for journals or conference proceedings be applied. The main
% Latex file contains several options as described in the Latex comment
% lines.
% 
% It is expected that these guidelines are implemented for papers already
% before they go into the first collaboration wide review. 
% 
% This template also contains the guidelines for how publications and
% conference reports should be written. 
% The symbols defined in \texttt{lhcb-symbols-def.tex} are compatible
% LHCb guidelines.
% 
% The front page should be adjusted according to what is
% written. Default versions are available for papers, conference reports
% and analysis notes. Just comment out what you require in the
% \texttt{main.tex} file.
% 
% This directory contains a file called \texttt{Makefile}.
% Typing \texttt{make} will apply all Latex and Bibtex commands 
% in the correct order to produce a pdf file of the document.
% The default Latex compliler is pdflatex, which requires figures 
% to be in pdf format. 
% To change to plain Latex, edit line 9 of \texttt{Makefile}.
% Typing \texttt{make clean} will remove all temporary files generated by (pdf)latex.



